\documentclass[a4paper,11.5pt,]{article}
\usepackage[utf8]{inputenc}
\usepackage[T1]{fontenc,url}
\usepackage[english,]{babel}
\usepackage{blindtext}
\usepackage{natbib}
\usepackage{gensymb}
\usepackage{amsmath}
\usepackage{changepage}
\usepackage{amssymb}
\usepackage{commath}
\usepackage{physics}
\usepackage{multicol}
\usepackage{float}
\usepackage{listings}
\usepackage{graphicx}
\usepackage{hyperref}
\usepackage{svg}
\usepackage{wrapfig}
\urlstyle{sf}


 \title{ \textbf{FYS 4150 - Computational Physics\\
 Project 1: Solving Poisson's equation in one dimension 
}}
 \date{\normalsize{4. September 2018} }
 \author{
 \textsc{\small{Maren Rasmussen}}\and \textsc{\small{Markus Leira Asprusten}}\and \textsc{\small{Metin San}}
 }
 
 
\begin{document}

\maketitle

\begin{center}
\textsc{Abstract}
\end{center}

\begin{adjustwidth}{7mm}{7mm}

This project involves solving the one-dimensional Possion equation with dirichlet boundary conditions using two different algorithms. The first method is the tridiagonal matrix algorithm while the second is the LU decomposition. The conclusion of the project is that a specialized version of the tridiagonal algorithm is much faster.

\end{adjustwidth}



\bigskip

\begin{center}
\textsc{1. Introduction}
\end{center}

\blindtext

\bigskip

\begin{center}
\textsc{1. Method}
\end{center}

\textbf{1.1 The Poisson equation} 

\blindtext

\end{document}



