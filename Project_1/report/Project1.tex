%setting document class
\documentclass[a4paper,10pt]{article}

%importing packages
\usepackage[utf8]{inputenc}
\usepackage[T1]{fontenc,url}
\usepackage[english,]{babel}
\usepackage{blindtext}
\usepackage{natbib}
\usepackage{gensymb}
\usepackage{amsmath}
\usepackage{changepage}
\usepackage{amssymb}
\usepackage{commath}
\usepackage{physics}
\usepackage{multicol}
\usepackage{float}
\usepackage{listings}
\usepackage{graphicx}
\usepackage{hyperref}
\usepackage{svg}
\usepackage{wrapfig}
\usepackage{fancyhdr}

%fixing the header 
\pagestyle{fancy}
\renewcommand{\headrulewidth}{0pt}
\fancyhf{}
\fancyhead[CO]{\textsc{asprusten \& rasmussen \& san }}
\fancyfoot[C]{\thepage}

%front page title/author setup
 \title{FYS 4150 - Computational Physics\\
 Project 1: Solving Poisson's equation in one dimension 
}
 \date{\normalsize{4. September 2018} }
 \author{
 \textsc{\small{Maren Rasmussen}}\and \textsc{\small{Markus Leira Asprusten}}\and \textsc{\small{Metin San}}
 }
 

 %starting the document
\begin{document}
\maketitle

\begin{center}
\textsc{Abstract}
\end{center}

\begin{adjustwidth}{7mm}{7mm}

This project involves solving the one-dimensional Possion equation with dirichlet boundary conditions using two different algorithms. The first method is the tridiagonal matrix algorithm while the second is the LU decomposition. The conclusion of the project is that a specialized version of the tridiagonal algorithm is much faster.

\end{adjustwidth}



\bigskip

\begin{center}
\textsc{1. Introduction}
\end{center}

\blindtext
\newpage


\begin{center}
\textsc{3. Algorithm}
\end{center}
In order to solve Poisson's equation we need to be able to solve the discretized set of equations involving the tridiagonal matrix $\textbf{A}$. We will tackle this problem through the implementation of two algorithms. The first is the Tridiagonal matrix algorithm, also known as the Thomas algorithm. The second is the LU-decomposition algorithm. 
\bigskip

3.1. \textbf{Tridiagonal Matrix Algorithm.} This algorithm is a simplified form of Gaussian elimination which can be used to solve tridiagonal systems of equations. A tridiagonal system of $n$ unknowns can be represented as 

\begin{equation}
a_i v_{i-1} + b_i v_i +c_i v_{i+1} = b_i,
\end{equation}
where $a_1 = c_1 = 0$. Or in matrix representation as $\textbf{A} \mathbf{v} = \mathbf{b}$. Written out in the $4 \cross 4$ case, this becomes
\begin{equation}
\begin{bmatrix}
b_1 & c_1 & 0 & 0 \\
a_2& b_2 & c_2 & 0 \\
0 & a_3 & b_3 & c_3 \\
0 & 0 & a_4 & b_4 
\end{bmatrix}
\begin{bmatrix}
v_1 \\
v_2 \\
v_3 \\
v_4
\end{bmatrix}
=
\begin{bmatrix}
f_1\\
f_2\\
f_3\\
f_4
\end{bmatrix}.
\end{equation}
The algorithm starts off by reducing the tridiagonal matrix $\textbf{A}$ to an upper tridiagonal matrix. This is achieved through Gaussian elimination. We want to get rid of the $a_i$ terms located on the lower secondary diagonal. We perform the following row reduction on both sides of the equation
\[
\begin{bmatrix}
b_1 & c_1 & 0 & 0 \\
a_2& b_2 & c_2 & 0 \\
0 & a_3 & b_3 & c_3 \\
0 & 0 & a_4 & b_4 
\end{bmatrix}
\xrightarrow{\text{II} - \frac{a_2}{b_1}\text{I}}
\begin{bmatrix}
b_1 & c_1 & 0 & 0 \\
0& \tilde{b_2}  & c_2 & 0 \\
0 & a_3 & b_3 & c_3 \\
0 & 0 & a_4 & b_4 
\end{bmatrix}, \qquad
\begin{bmatrix}
f_1\\
f_2\\
f_3\\
f_4
\end{bmatrix}
\xrightarrow{\text{II} - \frac{a_2}{b_1}\text{I}}
\begin{bmatrix}
f_1\\
\tilde{f_2}\\
f_3\\
f_4
\end{bmatrix}
\]
where $\tilde{b_2} = b_2 - a_2c_1/b_1$, and $\tilde{f_2} = f_2 - f_1a_2/b_1$, and II and I denotes the row 1 and 2 in the \textbf{A}. Similarly for the second row

\[
\begin{bmatrix}
b_1 & c_1 & 0 & 0 \\
0& \tilde{b_2}  & c_2 & 0 \\
0 & a_3 & b_3 & c_3 \\
0 & 0 & a_4 & b_4 
\end{bmatrix}
\xrightarrow{\text{III} - \frac{a_3}{b_2}\text{II}}
\begin{bmatrix}
b_1 & c_1 & 0 & 0 \\
0& \tilde{b_2}  & c_2 & 0 \\
0 & 0 & \tilde{b_3} & c_3 \\
0 & 0 & a_4 & b_4 
\end{bmatrix}, \qquad
\begin{bmatrix}
f_1\\
\tilde{f_2}\\
f_3\\
f_4
\end{bmatrix}
\xrightarrow{\text{III} - \frac{a_3}{b_2}\text{II}}
\begin{bmatrix}
f_1\\
\tilde{f_2}\\
\tilde{f_3}\\
f_4
\end{bmatrix}
\]
where $\tilde{b_3} = b_3 - a_3c_2/b_2$, and $\tilde{f_3} = f_3 - f_2a_3/b_2$. Finally we compute the last row reduction

\[
\begin{bmatrix}
b_1 & c_1 & 0 & 0 \\
0& \tilde{b_2}  & c_2 & 0 \\
0 & 0 & \tilde{b_3} & c_3 \\
0 & 0 & a_4 & b_4 
\end{bmatrix}
\xrightarrow{\text{IIII} - \frac{a_4}{b_3}\text{III}}
\begin{bmatrix}
b_1 & c_1 & 0 & 0 \\
0& \tilde{b_2}  & c_2 & 0 \\
0 & 0 & \tilde{b_3} & c_3 \\
0 & 0 & 0 & \tilde{b_4} 
\end{bmatrix}, \qquad
\begin{bmatrix}
f_1\\
\tilde{f_2}\\
\tilde{f_3}\\
f_4
\end{bmatrix}
\xrightarrow{\text{III} - \frac{a_3}{b_2}\text{II}}
\begin{bmatrix}
f_1\\
\tilde{f_2}\\
\tilde{f_3}\\
\tilde{f_4}
\end{bmatrix}
\]

where $\tilde{b_4} = b_4 - a_4c_3/b_3$, and $\tilde{f_4} = f_4 - f_3a_4/b_3$. 

\newpage

We are then left with the row reduced form of the set of equations $\tilde{\mathbf{A}}\mathbf{v} = \tilde{\mathbf{f}}$




\end{document}