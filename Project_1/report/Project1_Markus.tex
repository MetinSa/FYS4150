%setting document class
\documentclass[a4paper,10pt]{article}

%importing packages
\usepackage[utf8]{inputenc}
\usepackage[T1]{fontenc,url}
\usepackage[english,]{babel}
\usepackage{blindtext}
\usepackage{natbib}
\usepackage{gensymb}
\usepackage{amsmath}
\usepackage{changepage}
\usepackage{amssymb}
\usepackage{commath}
\usepackage{physics}
\usepackage{multicol}
\usepackage{float}
\usepackage{listings}
\usepackage{graphicx}
\usepackage{hyperref}
\usepackage{svg}
\usepackage{wrapfig}
\usepackage{fancyhdr}
\usepackage{color}

\definecolor{awesome}{rgb}{1.0, 0.13, 0.32}
\definecolor{mygray}{rgb}{0.5,0.5,0.5}
\definecolor{cadetgrey}{rgb}{0.57, 0.64, 0.69}

%listing customization
\lstset{ %
  backgroundcolor=\color{white},
  basicstyle=\footnotesize,        % the size of the fonts that are used for the code
  breakatwhitespace=true,         % sets if automatic breaks should only happen at whitespace
  breaklines=true,                 % sets automatic line breaking
  captionpos=b,                    % sets the caption-position to bottom
  commentstyle=\color{cadetgrey},    % comment style
  deletekeywords={...},            % if you want to delete keywords from the given language
  escapeinside={\%*}{*)},          % if you want to add LaTeX within your code
  extendedchars=true,              % lets you use non-ASCII characters; for 8-bits encodings only, does not work with UTF-8
  frame=single,	                   % adds a frame around the code
  keepspaces=true,                 % keeps spaces in text, useful for keeping indentation of code (possibly needs columns=flexible)
  keywordstyle=\color{awesome},       % keyword style
  language=c++,                 % the language of the code
  otherkeywords={...},           % if you want to add more keywords to the set
  rulecolor=\color{black},         % if not set, the frame-color may be changed on line-breaks within not-black text (e.g. comments (green here))
  showspaces=false,                % show spaces everywhere adding particular underscores; it overrides 'showstringspaces'
  showstringspaces=false,          % underline spaces within strings only
  showtabs=false,                  % show tabs within strings adding particular underscores
  stepnumber=2,                    % the step between two line-numbers. If it's 1, each line will be numbered
  stringstyle=\color{mymauve},     % string literal style
  tabsize=3,	                   % sets default tabsize to 2 spaces
}

%fixing the header 
\pagestyle{fancy}
\renewcommand{\headrulewidth}{0pt}
\fancyhf{}
\fancyhead[CO]{\textsc{asprusten \& rasmussen \& san }}
\fancyfoot[C]{\thepage}

%front page title/author setup
 \title{FYS 4150 - Computational Physics\\
 Project 1: Solving Poisson's equation in one dimension 
}
 \date{\normalsize{4. September 2018} }
 \author{
 \textsc{\small{Maren Rasmussen}}\and \textsc{\small{Markus Leira Asprusten}}\and \textsc{\small{Metin San}}
 }
 
\newcommand{\der}[2]{\frac{d #1}{d #2}}
\newcommand{\dder}[2]{\frac{d^2 #1}{d #2 ^2}}
\newcommand{\peder}[2]{\frac{\partial #1}{\partial #2}}


 %starting the document
\begin{document}
\maketitle
\begin{center}
\textsc{Abstract}
\end{center}

\begin{adjustwidth}{7mm}{7mm}


\end{adjustwidth}



\bigskip

\begin{center}
\textsc{1. Introduction}
\end{center}



\begin{center}
\textsc{2. Theoretical Background}

\end{center}
\begin{center}
\textsc{3. Algorithm \& Implementation }
\end{center}
\bigskip
3.2 \textbf{LU-decomposition} \\
Solving the linear algebra problem with an LU decomposition is relatively simple. By decomposing the matrix $\mathbf{\hat{A}}$ into the lower triangular matrix $\mathbf{\hat{L}}$ and the upper triangular matrix $\mathbf{\hat{U}}$, where all the diagonal elements in $\mathbf{\hat{L}}$ is 1, in such a way that $\mathbf{\hat{A}} = \mathbf{\hat{L}\hat{U}}$. We can then rewrite the linear algebra problem into

\begin{align}
	\mathbf{\hat{A}}\mathbf{\hat{u}} &= \mathbf{\hat{f}} \nonumber\\
	\mathbf{\hat{L}}\mathbf{\hat{U}}\mathbf{\hat{u}} &= \mathbf{\hat{f}} \nonumber\\
	\mathbf{\hat{U}}\mathbf{\hat{u}} &= \mathbf{\hat{L}}^{-1}\mathbf{\hat{f}} = \mathbf{\hat{y}} \nonumber\\
	\implies \mathbf{\hat{L}\hat{y}} = \mathbf{\hat{f}} \, &,\, \mathbf{\hat{U}\hat{u}} = \mathbf{\hat{y}}.
\end{align}
The problem is then to solve two equations, firstly for $\mathbf{\hat{y}}$ and lastly for $\mathbf{\hat{u}}$. Suppose the matrices have dimension ($n\times n$). The solution can then be found by iterating $n$-times for each equation, to a total of $2n$ iterations. The Armadillo library solves this problem simply with the \texttt{solve}-function.

\bigskip

\begin{center}
\textsc{4. Results}
\end{center}
\bigskip
4.5 \textbf{LU-decomposition} \\
As described is section \textit{3.2}, the \texttt{solve}-function is implemented in \texttt{problem\_e.cpp}, with the number of columns and rows in the square matrix, $n$, given as a command line input. The run times for this code can be found in table \ref{tab:LUres}. Generally speaking, we see clearly that the LU-decomposition has a longer computational time than the other algorithms to get the same results. Note also that the LU-solver would not run for $n = 10^5$, due to memory allocation error.


\begin{table}[]
\begin{tabular}{llll}
\hline
\hline
$n$ &General Algorithm  & Special Algorithm & LU-Decomposition \\
\hline
$10^1$ & $1.2\cdot 10^{-6}$ s & $1.2\cdot 10^{-6}$ s & $4.4\cdot 10^{-5}$ s\\
$10^2$ & $3.1\cdot 10^{-6}$ s & $2.9\cdot 10^{-6}$ s & $8.8\cdot 10^{-5}$ s\\
$10^3$ & $2.2\cdot 10^{-5}$ s & $2.2\cdot 10^{-5}$ s & $5.2\cdot 10^{-3}$ s\\
$10^4$ & $2.2\cdot 10^{-4}$ s & $2.3\cdot 10^{-4}$ s & $5.1\cdot 10^{-1}$ s\\
$10^5$ & $2.2\cdot 10^{-3}$ s & $2.2\cdot 10^{-3}$ s & N/A               \\
$10^6$ & $2.1\cdot 10^{-2}$ s & $2.1\cdot 10^{-2}$ s & N/A
\end{tabular}
\caption{Average time usage for 500 executions of the different algorithms, tested on Lenovo laptop running linux.}
\label{tab:LUres}
\end{table}
\bigskip
\begin{center}
\textsc{Discussion}
\end{center}

The relative error in figure \ref{fig:4} is decreasing steadily to about $n = 10^6$, after which it starts increasing. In other words, the error in the calculations is at its lowest at $n=10^6$. The reason for the increase in error with $n$ larger than this, might seem difficult to explain with an analytical mindset, but there is a lower limit to the precision of a number that a computer can represent. This means that that a larger $n$ will actually make the computations more imprecise as the differences calculated becomes smaller than the machine precision. \\

The difference in run time between the specialized algorithm discussed in section 4.2 and the general algorithm discussed in section 4.1 seems to depend heavily on the computer doing the operations. As seen in table \ref{tab:LUres}, the results from the Lenovo laptop has a minimal difference in rum time between the general and specialized algorithm, and the small differences there is, might be explained by random chance. The reason for this result is difficult to tell. As seen in table [METINS/MARENS RESULTATER], the results differ when running on another machine. These values are more as expected as the run times for the general algorithm is slower than for the specialized algorithm. The reason this is (and should be) the case, is that the specialized algorithm uses less FLOPS per iteration than the general algorithm.\\

The results seen in section 4.5 and table \ref{tab:LUres} shows that the LU-decomposition is noticeably slower than the two other algorithms discussed here. This can be explained by the number of elements that are needed to compute the LU-decomposition. The more specialized only need to to operations with 3 vector of length $n$, while the solution with LU-decompositions needs to do operations on matrices with dimensions $n\times n$. The problem with this can be seen in table \ref{tab:LUres}, where already at $n = 10^5$, the computer in use does not have enough memory to store all the numbers needed. On the other hand of this problem, the other algorithms exclusively solve a banded matrix, while the LU-decomposition can solve a more general linear algebra problem. So each method has its advantages and drawbacks.

\end{document}