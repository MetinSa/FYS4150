%setting document class
\documentclass[a4paper,10pt]{article}

%importing packages
\usepackage[utf8]{inputenc}
\usepackage[T1]{fontenc,url}
\usepackage[english,]{babel}
\usepackage{blindtext}
\usepackage{natbib}
\usepackage{gensymb}
\usepackage{amsmath}
\usepackage{changepage}
\usepackage{amssymb}
\usepackage{commath}
\usepackage{physics}
\usepackage{multicol}
\usepackage{float}
\usepackage{listings}
\usepackage{graphicx}
\usepackage{hyperref}
\usepackage{svg}
\usepackage{wrapfig}
\usepackage{fancyhdr}
\usepackage{color}

\definecolor{awesome}{rgb}{1.0, 0.13, 0.32}
\definecolor{mygray}{rgb}{0.5,0.5,0.5}
\definecolor{cadetgrey}{rgb}{0.57, 0.64, 0.69}

%listing customization
\lstset{ %
  backgroundcolor=\color{white},
  basicstyle=\footnotesize,        % the size of the fonts that are used for the code
  breakatwhitespace=true,         % sets if automatic breaks should only happen at whitespace
  breaklines=true,                 % sets automatic line breaking
  captionpos=b,                    % sets the caption-position to bottom
  commentstyle=\color{cadetgrey},    % comment style
  deletekeywords={...},            % if you want to delete keywords from the given language
  escapeinside={\%*}{*)},          % if you want to add LaTeX within your code
  extendedchars=true,              % lets you use non-ASCII characters; for 8-bits encodings only, does not work with UTF-8
  frame=single,	                   % adds a frame around the code
  keepspaces=true,                 % keeps spaces in text, useful for keeping indentation of code (possibly needs columns=flexible)
  keywordstyle=\color{awesome},       % keyword style
  language=c++,                 % the language of the code
  otherkeywords={...},           % if you want to add more keywords to the set
  rulecolor=\color{black},         % if not set, the frame-color may be changed on line-breaks within not-black text (e.g. comments (green here))
  showspaces=false,                % show spaces everywhere adding particular underscores; it overrides 'showstringspaces'
  showstringspaces=false,          % underline spaces within strings only
  showtabs=false,                  % show tabs within strings adding particular underscores
  stepnumber=2,                    % the step between two line-numbers. If it's 1, each line will be numbered
  stringstyle=\color{mymauve},     % string literal style
  tabsize=3,	                   % sets default tabsize to 2 spaces
}

%fixing the header 
\pagestyle{fancy}
\renewcommand{\headrulewidth}{0pt}
\fancyhf{}
\fancyhead[CO]{\textsc{asprusten \& rasmussen \& san }}
\fancyfoot[C]{\thepage}

%front page title/author setup
 \title{FYS 4150 - Computational Physics\\
 Project 1: Solving Poisson's equation in one dimension 
}
 \date{\normalsize{4. September 2018} }
 \author{
 \textsc{\small{Maren Rasmussen}}\and \textsc{\small{Markus Leira Asprusten}}\and \textsc{\small{Metin San}}
 }
 
\newcommand{\der}[2]{\frac{d #1}{d #2}}
\newcommand{\dder}[2]{\frac{d^2 #1}{d #2 ^2}}
\newcommand{\peder}[2]{\frac{\partial #1}{\partial #2}}


 %starting the document
\begin{document}
\maketitle
\begin{center}
\textsc{Abstract}
\end{center}

\begin{adjustwidth}{7mm}{7mm}


\end{adjustwidth}



\bigskip

\begin{center}
\textsc{1. Introduction}
\end{center}

\newpage


\begin{center}
\textsc{2. Theoretical Background}

\end{center}
\begin{center}
\textsc{3. Algorithm \& Implementation }
\end{center}
\bigskip
3.2 \textbf{LU-decomposition} \\
Solving the linear algebra problem with an LU decomposition is relatively simple. By decomposing the matrix $\mathbf{\hat{A}}$ into the lower triangular matrix $\mathbf{\hat{L}}$ and the upper triangular matrix $\mathbf{\hat{U}}$, where all the diagonal elements in $\mathbf{\hat{L}}$ is 1, in such a way that $\mathbf{\hat{A}} = \mathbf{\hat{L}\hat{U}}$. We can then rewrite the linear algebra problem into

\begin{align}
	\mathbf{\hat{A}}\mathbf{\hat{u}} &= \mathbf{\hat{f}} \nonumber\\
	\mathbf{\hat{L}}\mathbf{\hat{U}}\mathbf{\hat{u}} &= \mathbf{\hat{f}} \nonumber\\
	\mathbf{\hat{U}}\mathbf{\hat{u}} &= \mathbf{\hat{L}}^{-1}\mathbf{\hat{f}} = \mathbf{\hat{y}} \nonumber\\
	\implies \mathbf{\hat{L}\hat{y}} = \mathbf{\hat{f}} \, &,\, \mathbf{\hat{U}\hat{u}} = \mathbf{\hat{y}}.
\end{align}
The problem is then to solve two equations, firstly for $\mathbf{\hat{y}}$ and lastly for $\mathbf{\hat{u}}$. Suppose the matrices have dimension ($n\times n$). The solution can then be found by iterating $n$-times for each equation, to a total of $2n$ iterations. The Armadillo library solves this problem simply with the \texttt{solve}-function.

\bigskip

\begin{center}
\textsc{4. Results}
\end{center}
\bigskip
4.5 \textbf{LU-decomposition} \\



\begin{table}[]  \label{fig:5}
\begin{tabular}{llll}
\hline
n &General Algorithm  & Special Algorithm & LU-Decomposition \\
\hline
10 & 0.030738           & 0.023905          &                  \\
100 & 0.028017           & 0.024106          &                  \\
1000 & 0.028026           & 0.024124          &                  \\
10 000 & 0.028299           & 0.023557          &                  \\
100 000 & 0.028742           & 0.024067          &                  \\
\end{tabular}
\end{table}


\end{document}